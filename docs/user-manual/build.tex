\section{Installing the \pate{} Verifier}
\label{sec:build-pate-verif}

\subsection{Recommended: Docker Container Image}

We recommend using \pate{} via Docker.

If available, load the Docker image file from a file using:
\begin{verbatim}
  docker load -i /path/to/pate.tar.gz
\end{verbatim}

Otherwise, build the Docker image from the \pate{} source repo.
Building \pate{} is a memory-intensive process.
We build the Docker image in an environment with 16 GB of RAM available to Docker.
Build the image using the following commands:

\begin{verbatim}
  git clone git@github.com:GaloisInc/pate.git
  cd pate
  git submodule update --init
  docker build --platform linux/amd64 . -t pate
\end{verbatim}

\pate{} may subsequently be used via Docker, such as:

\begin{verbatim}
  docker run --rm -it \
    --platform linux/amd64 \
    -v "$(pwd)"/tests/integration/packet:/target \
    pate \
    --original /target/packet.original.exe \
    --patched /target/packet.patched.exe \
    -s parse_packet
\end{verbatim}

Please see later sections for detailed usage information.

\subsection{Alternative: Building from Source}

Alternatively, \pate{} may be compiled from source and run on a host system.
The \pate{} tool is written in Haskell and requires the GHC compiler\footnote{\url{https://www.haskell.org/ghc/}} and cabal\footnote{\url{https://www.haskell.org/cabal/}} to build.

The current version of \pate{} is developed using GHC version 9.6.2 and cabal version 3.10.2.0.
Additionally, the verifier requires an SMT solver to be available in \texttt{PATH}.
We recommend \texttt{yices}.
The \texttt{z3} and \texttt{cvc4} solvers may also work but are not regularly tested with \pate{}.

Building from source can be accomplished as follows:
\begin{verbatim}
  git clone git@github.com:GaloisInc/pate.git
  cd pate
  git submodule update --init
  cp cabal.project.dist cabal.project
  cabal configure pkg:pate
  ./pate.sh --help
\end{verbatim}

Note that \texttt{./pate.sh} should be used to build \pate{} (not \texttt{cabal build} or similar).

Once built, invoke \pate{} locally using the \texttt{pate.sh} script.

%%% Local Variables:
%%% mode: latex
%%% TeX-master: "user-manual"
%%% End:
