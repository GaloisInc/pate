\section{Proving Equivalence}\label{sec:equivalence}

The primary feature of \TOOL{} is statically verifying that two binaries, and original binary and a patched binary, have the same observable behaviors. %
It does this by constructing a compositional proof of equivalence between the two binaries. %
The proof aligns the loop-free and call-free code sequences in the binaries and computes a sufficient set of pre- and post-conditions for each pair of code sequences to exhibit identical behavior. %
Each of the post-conditions in the proof are \emph{proof obligations}; if all of the proof obligations are satisfied, the two programs have the same observable behaviors. %
\TOOL{} then traverses the proof to automatically discharge all of the proof obligations using an SMT solver. %

In practice, we do not expect the two programs to be equivalent, as the program was patched for a reason (i.e., to change its behavior). %
If the two binaries can exhibit different behaviors, \TOOL{} identifies the part of the program where the difference can manifest, analyzes the counterexample provided by the SMT solver that exhibits the different behavior, and generates a \emph{differential summary} that explains the conditions under which the difference in behavior is observable. %
A differential summary is a logical formula in terms of function arguments and global variables that precisely describes the conditions under which the two binaries exhibit different behavior. %
Formally, if one \emph{assumes} that the differential summary is true, at the beginning of a function, the patched program will exhibit different behavior than the original program. %


\begin{lstlisting}[caption={An example differential summary}, label=fig:differential-summary]
let
v376575 = select cInitMem@372468:a 0
v423090 = select v376575 (bvSum R0:bv 0x4:[32])
v423130 = select v376575 (bvSum R1:bv 0x4:[32])
v423111 = select v376575 (bvSum R0:bv 0x2:[32])
v423104 = select v376575 (bvSum R0:bv 0x3:[32])
v423125 = bvSum (bvZext 16 v423111) (bvShl (bvZext 16 v423104) 0x8:[16])
v432396 = bvSlt v423125 0x0:[16]
v423131 = eq 0x0:[8] v423130
v432397 = and (not (eq 0x0:[2] (bvSelect 2 2 v423090))) v423131 v432396
 in not v432397
\end{lstlisting}

An example differential summary is shown in \cref{fig:differential-summary}. %
This formula describes a relationship between a number of values read from pointers passed via arguments in \lstinline{R0} and \lstinline{R1}. %
The differential summaries are available for inspection both in the interactive proof explorer UI (see \cref{sec:options} for details) and in the JSON reports generated by the verifier. %

\subsection{Code Alignment}

To construct the equivalence proof between an original and patched binary, \TOOL{} breaks each binary up into loop-free and call-free sequences of instructions. %
It then \emph{aligns} the corresponding code sequences in the original and patched binary. %
If this alignment is unsuccessful, the proof will ultimately fail, and will further more fail to produce useful differential diagnostics. %
The verifier is robust to many of the methods for inserting micropatches (e.g., redirecting execution to a patch location and jumping back). %
Trouble typically arises in the presence of larger code motion that can arise from recompilation, rather than micropatching. %
Additionally, any movement of mutable \emph{data} (e.g., shifting the contents of the \lstinline{.data} or \lstinline{.bss} sections) is likely to cause catastrophic failures, as the verifier will be unable to find a meaningful comparison between the original and patched binaries. %
If a patch requires larger scale changes that do not fit the micropatching model, \TOOL{} provides an additional verification mode that can deal with some types of larger changes, which is described in \cref{sec:inlining}.
