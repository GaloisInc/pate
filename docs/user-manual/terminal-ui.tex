\section{The \pate{} Terminal UI}
\label{sec:terminal-ui}

Once \pate{} has been invoked at the command line (see Section~\ref{sec:run-pate-verif}), the user is presented with an interactive terminal user interface.

Internally, \pate{} maintains a tree representing the state of the analysis.
The interactive interface allows users to inspect the tree by selecting from a list of options based on the users current ``location'' in the tree.
The user makes a selection by entering a number representing the node of interest and hitting enter, or by entering a command.

\subsection{Example Usage}

Launch \pate{} on the binaries in the \texttt{tests/integration/packet/} directory with:

\begin{verbatim}
   docker run --rm -it \
     --platform linux/amd64 \
     -v "$(pwd)"/tests/integration/packet:/target \
     pate \
     --original /target/packet.original.exe \
     --patched /target/packet.patched.exe \
     -s parse_packet
 \end{verbatim}

Once launched, \pate{} presents the user is presented with a list of entry points and makes a selection:
\begin{lstlisting}
  Choose Entry Point
  0: Function Entry "_start" (segment1+0x435)
  1: Function Entry "parse_packet" (segment1+0x590)
  ?> |\colorbox{yellow}{1}|
\end{lstlisting}

We indicate user input with a yellow highlight.
Here the user enters \texttt{1}, selecting \texttt{parse\_packet}.
This starts \pate{}'s analysis at this point, and the user sees output of:

\begin{lstlisting}
..
0: Function Entry "parse_packet" (segment1+0x590) (User Request)
1: segment1+0x5c0 [ via: "parse_packet" (segment1+0x590) ] 
   (Widening Equivalence Domains)
2: segment1+0x600 [ via: "parse_packet" (segment1+0x590) ] 
   (Widening Equivalence Domains)
3: Return "parse_packet" (segment1+0x590) 
   (Widening Equivalence Domains)
4: segment1+0x63c [ via: "parse_packet" (segment1+0x590) ] 
   (Widening Equivalence Domains)
5: segment1+0x658 [ via: "parse_packet" (segment1+0x590) ] 
   (Widening Equivalence Domains)
6: segment1+0x698 [ via: "parse_packet" (segment1+0x590) ] 
   (Widening Equivalence Domains)
7: segment1+0x6d8 [ via: "parse_packet" (segment1+0x590) ] 
   (Widening Equivalence Domains)
8: Function Entry "check_crc" (segment1+0x530) 
   [ via: "parse_packet" (segment1+0x6ec) ] 
   (Widening Equivalence Domains)
9: Return "check_crc" (segment1+0x530) 
   [ via: "parse_packet" (segment1+0x6ec) ] 
   (Widening Equivalence Domains)
10: segment1+0x6ec [ via: "parse_packet" (segment1+0x590) ] 
    (Widening Equivalence Domains)
Handle observable difference:
0: Emit warning and continue 
1: Assert difference is infeasible (defer proof) 
2: Assert difference is infeasible (prove immediately) 
3: Assume difference is infeasible 
4: Avoid difference with equivalence condition 
5: Avoid difference with path-sensitive equivalence condition
?> |\colorbox{yellow}{4}|
\end{lstlisting}

\pate{} represents the analysis as a tree that can be navigated by the user.
The top level of the interactive process (reachable via the \texttt{top} command) is a list of all analysis steps that were taken, starting from the selected entry point.
Each pair of address and calling contexts defines a unique toplevel proof ``node.''
A given address and context may appear multiple times in the toplevel list, corresponding to each individual time that the address/context pair was analyzed.
The latest (highest-numbered) entry corresponds to the most recent analysis of an address/context.

Each entry is associated with an equivalence domain: a set of locations (registers, stack slots and memory addresses) that are potentially not equal at this point.
Locations outside this set have been proven to be equal (ignoring skipped functions).
The analysis takes the equivalence domain of an address/context pair and computes an equivalence domain for each possible exit (according to the semantics of the block).

At this point in the example, \pate{} is asking how a detected observable difference should be handled.
The user selects \texttt{4} to capture the difference in the equivalence condition, and \pate{} continues its analysis:

\begin{lstlisting}
0: Function Entry "parse_packet" (segment1+0x590) (User Request)
1: segment1+0x5c0 [ via: "parse_packet" (segment1+0x590) ] 
   (Widening Equivalence Domains)
2: segment1+0x600 [ via: "parse_packet" (segment1+0x590) ] 
   (Widening Equivalence Domains)
3: Return "parse_packet" (segment1+0x590) 
   (Widening Equivalence Domains)
4: segment1+0x63c [ via: "parse_packet" (segment1+0x590) ] 
   (Widening Equivalence Domains)
5: segment1+0x658 [ via: "parse_packet" (segment1+0x590) ] 
   (Widening Equivalence Domains)
6: segment1+0x698 [ via: "parse_packet" (segment1+0x590) ] 
   (Widening Equivalence Domains)
7: segment1+0x6d8 [ via: "parse_packet" (segment1+0x590) ] 
   (Widening Equivalence Domains)
8: Function Entry "check_crc" (segment1+0x530) 
   [ via: "parse_packet" (segment1+0x6ec) ] 
   (Widening Equivalence Domains)
9: Return "check_crc" (segment1+0x530) 
   [ via: "parse_packet" (segment1+0x6ec) ] 
   (Widening Equivalence Domains)
10: segment1+0x6ec [ via: "parse_packet" (segment1+0x590) ] 
    (Widening Equivalence Domains)
11: segment1+0x71c [ via: "parse_packet" (segment1+0x590) ] 
    (Widening Equivalence Domains)
12: segment1+0x71c [ via: "parse_packet" (segment1+0x590) ] 
    (Re-checking Block Exits)
13: segment1+0x6ec [ via: "parse_packet" (segment1+0x590) ] 
    (Propagating Conditions)
14: segment1+0x6ec [ via: "parse_packet" (segment1+0x590) ] 
    (Re-checking Block Exits)
15: Return "check_crc" (segment1+0x530) [ via: "parse_packet" (segment1+0x6ec) ] 
    (Propagating Conditions)
16: segment1+0x71c [ via: "parse_packet" (segment1+0x590) ] 
    (Widening Equivalence Domains)
17: Return "parse_packet" (segment1+0x590) (Widening Equivalence Domains)
18: segment1+0x758 [ via: "parse_packet" (segment1+0x590) ] 
    (Widening Equivalence Domains)
19: Function Entry "parse_data" (segment1+0x558) 
    [ via: "parse_packet" (segment1+0x774) ] 
    (Widening Equivalence Domains)
20: segment1+0x584 [ via: "parse_data" (segment1+0x558) 
      <- "parse_packet" (segment1+0x774) ] 
    (Widening Equivalence Domains)
21: Return "parse_data" (segment1+0x558) 
    [ via: "parse_packet" (segment1+0x774) ] 
    (Widening Equivalence Domains)
22: segment1+0x774 [ via: "parse_packet" (segment1+0x590) ] 
    (Widening Equivalence Domains)
23: Return "parse_packet" (segment1+0x590) 
    (Widening Equivalence Domains)
24: segment1+0x5c0 [ via: "parse_packet" (segment1+0x590) ] 
    (Re-checking Equivalence Domains)
25: Return "parse_packet" (segment1+0x590) 
    (Widening Equivalence Domains)
26: segment1+0x600 [ via: "parse_packet" (segment1+0x590) ] 
    (Re-checking Equivalence Domains)
27: segment1+0x658 [ via: "parse_packet" (segment1+0x590) ] 
    (Re-checking Equivalence Domains)
28: segment1+0x698 [ via: "parse_packet" (segment1+0x590) ] 
    (Re-checking Equivalence Domains)
Continue verification?
0: Finish and view final result 
1: Restart from entry point 
2: Handle pending refinements 
?> |\colorbox{yellow}{0}|
\end{lstlisting}

The user selects \texttt{0} to finish and view the final result.

The \pate{} analysis tree can be navigated by the user with \texttt{top} to move to the top of the tree, numbers to navigate ``into'' nodes, \texttt{up} to move ``up'' a node, and \texttt{ls} to re-display the nodes available at a current level.

For example, to inspect the analysis results in the running example, the user may provide input as follows to view the equivalence condition(s):

\begin{lstlisting}
  ...
  30: Final Result
  > |\colorbox{yellow}{30}|
  Final Result
  0: Assumed Equivalence Conditions
  1:   segment1+0x6ec [ via: "parse_packet" (segment1+0x590) ]
  2:   segment1+0x71c [ via: "parse_packet" (segment1+0x590) ]
  3: Binaries are conditionally, observably equivalent
  4: Toplevel Result
  > |\colorbox{yellow}{2}|
  segment1+0x71c [ via: "parse_packet" (segment1+0x590) ]
  0: ------
  1:   original
  2:   patched
  3: (Predicate) let -- segment1+0x734.. in not (and v67142 (not v67145))
  4: With condition assumed
  5:   Event Trace: segment1+0x71c .. segment1+0x754
  6: With negation assumed
  7:   Event Trace: segment1+0x71c .. segment1+0x754
  8: Simplified Predicate
  > |\colorbox{yellow}{3}|
  let -- segment1+0x734
    v67145 = eq 0x0:[8] (select (select cInitMemBytes@66997:a 0) 0x11044:[32])
    -- segment1+0x734
    v67142 = eq 0x80:[8] (select (select cInitMemBytes@67121:a 0) 0x11045:[32])
 in not (and v67142 (not v67145))
\end{lstlisting}

This is a formal representation of an equivalence condition calculated by \pate{}, which is the result of the user
choosing to ``avoid'' the detected observable difference. The condition is a predicate over the (symbolic) original and patched program
states starting at a given address/context (in this case, starting at \texttt{0x71c} within the \texttt{parse\_packet} function).
In the binary ninja plugin (see~\ref{sec:binary-ninja-ui}) this predicate is post-processed to be more human-readable.

Continuing on, the user may view, for example, an example trace showing where the equivalence condition above does not hold:

\begin{lstlisting}
  > |\colorbox{yellow}{up}|
  ...
  4: With condition assumed
  5:   Event Trace: segment1+0x71c .. segment1+0x754
  6: With negation assumed
  7:   Event Trace: segment1+0x71c .. segment1+0x754
  ...
  > |\colorbox{yellow}{7}|
  == Initial Original Registers ==
  pc <- 0x71c:[32]
  r0 <- 0x0:[32]
  r1 <- 0x1:[32]
  r11 <- Stack Slot: -8
  r13 <- Stack Slot: -24
  r14 <- 0x71c:[32]
  r2 <- 0x1:[32]
  ...
  == Original sequence ==
  ...
  (segment1+0x724)
    Read 0x11044:[32] -> 0x80:[8] 
    r0 <- 0x80:[32]
  (segment1+0x728) Write Stack Slot: -21 <- 0x80:[8] 
  ...
  (segment1+0x734)
    Read 0x11045:[32] -> 0x80:[8] 
    r0 <- 0x80:[32]
  ...
  (segment1+0x748)
  Read Stack Slot: -21 -> 0x80:[8] 
  r1 <- 0x80:[32]
  ...
  == Initial Patched Registers ==
  pc <- 0x71c:[32]
  r0 <- 0x0:[32]
  r1 <- 0x1:[32]
  r11 <- Stack Slot: -8
  r13 <- Stack Slot: -24
  r14 <- 0x71c:[32]
  r2 <- 0x1:[32]
  ...
  == Patched sequence ==
  ...
  (segment1+0x724)
    Read 0x11044:[32] -> 0x80:[8] 
    r0 <- 0x80:[32]
  (segment1+0x728) Write Stack Slot: -21 <- 0x80:[8] 
  ...
  (segment1+0x734)
    Read 0x11045:[32] -> 0x80:[8] 
    r0 <- 0x80:[32]
  ...
  (segment1+0x744) Write Stack Slot: -21 <- 0x0:[8] 
  (segment1+0x748)
    Read Stack Slot: -21 -> 0x0:[8] 
    r1 <- 0x0:[32]
  ...
\end{lstlisting}

Here \pate{} has constructed a trace where memory addresses \texttt{0x11044} and \texttt{0x11045} are both assumed
to contain \texttt{0x80} (negating the equivalence condition), and shown that this leads to a different value
assigned to stack slot \texttt{-21}. This slot is provided as input to a subsequent call to \texttt{printf} and is 
therefore considered an observable difference.

See the following subsections for details about how to interpret and interact with the terminal user interface.

\subsection{Status Indicators}

The \emph{prompt} indicates the status of the current node as follows:
\begin{itemize}
\item \texttt{*>} current node still has some active task running
\item \texttt{?>} current node requires user input
\item \texttt{!>} current node has raised a warning
\item \texttt{x>} current node has raised an error
\item \texttt{>} current node, and all sub-nodes, have finished processing
\end{itemize}

Similar to the prompt, nodes may be printed with a suffix that indicates some additional status as follows:
\begin{itemize}
\item \texttt{(*)} node still has some active task running
\item \texttt{(?)} node requires user input
\item \texttt{(!)} node has raised a warning
\item \texttt{(x)} node has raised an error
\end{itemize}
A status suffix indicates that the node, or some sub-node, has the given status.
e.g. at the toplevel the prompt \texttt{x>} indicates that an error was thrown during some block analysis, while the corresponding node for the block will have a \texttt{(x)} suffix.

\subsection{Navigation Commands}

\begin{itemize}
\item \texttt{\#} - navigate to a node, printing its contents
\item \texttt{up} - navigate up one tree level
\item \texttt{top} - navigate to the toplevel
\item \texttt{goto\_err} - navigate to the first leaf node with an error status
\item \texttt{next} - navigate to the highest-numbered node at the current level
\end{itemize}

\subsection{Diagnostic Commands}

\begin{itemize}
\item \texttt{status} - print the status of the current node
\item \texttt{full\_status} - print the status of the current node, without truncating the output
\item \texttt{ls} - print the list of nodes at the current level
\item \texttt{wait} - wait at the current level for more results. Exits when the node finishes, or
the user provides any input
\end{itemize}
When the prompt is \texttt{?>}, the verifier is waiting for input at some sub-node.
To select an option, simply navigate (i.e. by entering \texttt{\#}) to the desired choice.
For example, \texttt{goto\_prompt} - navigate to the first leaf node waiting for user input.


%%% Local Variables:
%%% mode: latex
%%% TeX-master: "user-manual"
%%% End:
